\documentclass[]{article}
\usepackage{lmodern}
\usepackage{amssymb,amsmath}
\usepackage{ifxetex,ifluatex}
\usepackage{fixltx2e} % provides \textsubscript
\ifnum 0\ifxetex 1\fi\ifluatex 1\fi=0 % if pdftex
  \usepackage[T1]{fontenc}
  \usepackage[utf8]{inputenc}
\else % if luatex or xelatex
  \ifxetex
    \usepackage{mathspec}
  \else
    \usepackage{fontspec}
  \fi
  \defaultfontfeatures{Ligatures=TeX,Scale=MatchLowercase}
\fi
% use upquote if available, for straight quotes in verbatim environments
\IfFileExists{upquote.sty}{\usepackage{upquote}}{}
% use microtype if available
\IfFileExists{microtype.sty}{%
\usepackage{microtype}
\UseMicrotypeSet[protrusion]{basicmath} % disable protrusion for tt fonts
}{}
\usepackage[margin=1in]{geometry}
\usepackage{hyperref}
\hypersetup{unicode=true,
            pdftitle={Main\_function.R},
            pdfauthor={USER},
            pdfborder={0 0 0},
            breaklinks=true}
\urlstyle{same}  % don't use monospace font for urls
\usepackage{color}
\usepackage{fancyvrb}
\newcommand{\VerbBar}{|}
\newcommand{\VERB}{\Verb[commandchars=\\\{\}]}
\DefineVerbatimEnvironment{Highlighting}{Verbatim}{commandchars=\\\{\}}
% Add ',fontsize=\small' for more characters per line
\usepackage{framed}
\definecolor{shadecolor}{RGB}{248,248,248}
\newenvironment{Shaded}{\begin{snugshade}}{\end{snugshade}}
\newcommand{\KeywordTok}[1]{\textcolor[rgb]{0.13,0.29,0.53}{\textbf{#1}}}
\newcommand{\DataTypeTok}[1]{\textcolor[rgb]{0.13,0.29,0.53}{#1}}
\newcommand{\DecValTok}[1]{\textcolor[rgb]{0.00,0.00,0.81}{#1}}
\newcommand{\BaseNTok}[1]{\textcolor[rgb]{0.00,0.00,0.81}{#1}}
\newcommand{\FloatTok}[1]{\textcolor[rgb]{0.00,0.00,0.81}{#1}}
\newcommand{\ConstantTok}[1]{\textcolor[rgb]{0.00,0.00,0.00}{#1}}
\newcommand{\CharTok}[1]{\textcolor[rgb]{0.31,0.60,0.02}{#1}}
\newcommand{\SpecialCharTok}[1]{\textcolor[rgb]{0.00,0.00,0.00}{#1}}
\newcommand{\StringTok}[1]{\textcolor[rgb]{0.31,0.60,0.02}{#1}}
\newcommand{\VerbatimStringTok}[1]{\textcolor[rgb]{0.31,0.60,0.02}{#1}}
\newcommand{\SpecialStringTok}[1]{\textcolor[rgb]{0.31,0.60,0.02}{#1}}
\newcommand{\ImportTok}[1]{#1}
\newcommand{\CommentTok}[1]{\textcolor[rgb]{0.56,0.35,0.01}{\textit{#1}}}
\newcommand{\DocumentationTok}[1]{\textcolor[rgb]{0.56,0.35,0.01}{\textbf{\textit{#1}}}}
\newcommand{\AnnotationTok}[1]{\textcolor[rgb]{0.56,0.35,0.01}{\textbf{\textit{#1}}}}
\newcommand{\CommentVarTok}[1]{\textcolor[rgb]{0.56,0.35,0.01}{\textbf{\textit{#1}}}}
\newcommand{\OtherTok}[1]{\textcolor[rgb]{0.56,0.35,0.01}{#1}}
\newcommand{\FunctionTok}[1]{\textcolor[rgb]{0.00,0.00,0.00}{#1}}
\newcommand{\VariableTok}[1]{\textcolor[rgb]{0.00,0.00,0.00}{#1}}
\newcommand{\ControlFlowTok}[1]{\textcolor[rgb]{0.13,0.29,0.53}{\textbf{#1}}}
\newcommand{\OperatorTok}[1]{\textcolor[rgb]{0.81,0.36,0.00}{\textbf{#1}}}
\newcommand{\BuiltInTok}[1]{#1}
\newcommand{\ExtensionTok}[1]{#1}
\newcommand{\PreprocessorTok}[1]{\textcolor[rgb]{0.56,0.35,0.01}{\textit{#1}}}
\newcommand{\AttributeTok}[1]{\textcolor[rgb]{0.77,0.63,0.00}{#1}}
\newcommand{\RegionMarkerTok}[1]{#1}
\newcommand{\InformationTok}[1]{\textcolor[rgb]{0.56,0.35,0.01}{\textbf{\textit{#1}}}}
\newcommand{\WarningTok}[1]{\textcolor[rgb]{0.56,0.35,0.01}{\textbf{\textit{#1}}}}
\newcommand{\AlertTok}[1]{\textcolor[rgb]{0.94,0.16,0.16}{#1}}
\newcommand{\ErrorTok}[1]{\textcolor[rgb]{0.64,0.00,0.00}{\textbf{#1}}}
\newcommand{\NormalTok}[1]{#1}
\usepackage{graphicx,grffile}
\makeatletter
\def\maxwidth{\ifdim\Gin@nat@width>\linewidth\linewidth\else\Gin@nat@width\fi}
\def\maxheight{\ifdim\Gin@nat@height>\textheight\textheight\else\Gin@nat@height\fi}
\makeatother
% Scale images if necessary, so that they will not overflow the page
% margins by default, and it is still possible to overwrite the defaults
% using explicit options in \includegraphics[width, height, ...]{}
\setkeys{Gin}{width=\maxwidth,height=\maxheight,keepaspectratio}
\IfFileExists{parskip.sty}{%
\usepackage{parskip}
}{% else
\setlength{\parindent}{0pt}
\setlength{\parskip}{6pt plus 2pt minus 1pt}
}
\setlength{\emergencystretch}{3em}  % prevent overfull lines
\providecommand{\tightlist}{%
  \setlength{\itemsep}{0pt}\setlength{\parskip}{0pt}}
\setcounter{secnumdepth}{0}
% Redefines (sub)paragraphs to behave more like sections
\ifx\paragraph\undefined\else
\let\oldparagraph\paragraph
\renewcommand{\paragraph}[1]{\oldparagraph{#1}\mbox{}}
\fi
\ifx\subparagraph\undefined\else
\let\oldsubparagraph\subparagraph
\renewcommand{\subparagraph}[1]{\oldsubparagraph{#1}\mbox{}}
\fi

%%% Use protect on footnotes to avoid problems with footnotes in titles
\let\rmarkdownfootnote\footnote%
\def\footnote{\protect\rmarkdownfootnote}

%%% Change title format to be more compact
\usepackage{titling}

% Create subtitle command for use in maketitle
\newcommand{\subtitle}[1]{
  \posttitle{
    \begin{center}\large#1\end{center}
    }
}

\setlength{\droptitle}{-2em}

  \title{Main\_function.R}
    \pretitle{\vspace{\droptitle}\centering\huge}
  \posttitle{\par}
    \author{USER}
    \preauthor{\centering\large\emph}
  \postauthor{\par}
      \predate{\centering\large\emph}
  \postdate{\par}
    \date{Tue Mar 10 19:18:43 2020}


\begin{document}
\maketitle

iNEXT 4 steps

\code{iNEXT.4steps}:a complete (random sampling) biological analysis
combined with four parts:\cr
Step1: Sample Completeness.\cr
Step2: Interpolation and Extrapolation.\cr
Step3: Asymptotic diversity.\cr
Step4: Evenness.\cr
@param data a matrix/data.frame of species abundances/incidences
data.\cr
Type (1) abundance data: When there are N assemblages, the observed
species abundances should be arranged as a species (in rows) by
assemblage (in columns) matrix. The first row (including N entries)
lists the assemblage labels or site names for the N assemblages.\cr
Type (2) incidence data: The data input format for incidence data must
be raw detection/non-detection data. That is, data for each
community/assemblage consist of a species-by-sampling-unit matrix. Users
must first merge multiple-community data by species identity to obtain a
pooled list of species; then the rows of the input data refer to this
pooled list. \cr
@param datatype data type of input data: individual-based abundance data
(\code{datatype = "abundance"}), sampling-unit-based incidence
frequencies data (\code{datatype = "incidence_freq"}) or species by
sampling-units incidence matrix (\code{datatype = "incidence_raw"}).\cr
@param size a vector of nonnegative integers specifying the sample sizes
for which diversity estimates will be calculated. If \code{NULL}, the
diversity estimates will be calculated for those sample sizes determined
by the specified/default \code{endpoint} and \code{knot}. \cr
(setting only for \code{step2}).\cr
@param endpoint an interger specifying the endpoint for rarefaction and
extrapolation range. If \code{NULL}, \code{endpoint} = double of the
maximum reference sample size. It will be ignored if \code{size} is
given. \cr
(setting only for \code{step2}).\cr
@param knots an integer specifying the number of knot between 1 and the
\code{endpoint}, default is 40.\cr
(setting only for \code{step2}).\cr
@param se a logical variable to calculate the bootstrap standard error
and confidence interval of a level specified by conf, default is
\code{TRUE}.\cr
(setting only for \code{step2}).\cr
@param conf a positive number \textless{} 1 specifying the level of
confidence interval, default is 0.95.\cr
(setting only for \code{step2}).\cr
@param nboot an integer specifying the number of bootstrap replications,
default is 50.\cr
(setting only for \code{step2}).\cr
@param details a logical variable to determine whether do you want to
print out the detailed value of 4 plots, default is \code{FALSE}.\cr
@import devtools @import iNEXT @import ggplot2 @import reshape2 @import
dplyr @import ggpubr @importFrom stats rbinom @importFrom stats
rmultinom @importFrom stats sd @importFrom stats qnorm @return a list of
two of objects: \cr\cr
\code{$summary} individual summary of 4 steps of data. \cr\cr
\code{$figure} 5 figures of analysis process. \cr\cr
\code{$details} the information for generating \code{figure}. \cr
if you nees it, you should key in \code{details = TRUE}. \cr\cr
@examples \donttest{
## Type (1) example for abundance based data (data.frame)
Ex.1
data(spider)
out1 <- iNEXT.4steps(data = spider, datatype = "abundance")
out1
## Type (2) example for incidence based data (list of data.frame)
Ex.2
data(woody_incid)
out2 <- iNEXT.4steps(data = woody_incid[,c(1,4)], datatype = "incidence_freq")
out2
} @references Chao,A., Y.Kubota, D.Zelený, C.-H.Chiu, C.-F.Li,
B.Kusumoto, M.Yasuhara, S.Thorn, C.-L.Wei, M.J.Costello, and
R.K.olwell(2020). Quantifying sample completeness and comparing
diversities among assemblages. Ecological Research. @export

\begin{Shaded}
\begin{Highlighting}[]
\NormalTok{iNEXT.4steps <-}\StringTok{ }\ControlFlowTok{function}\NormalTok{(data, }\DataTypeTok{datatype=}\StringTok{"abundance"}\NormalTok{, }\DataTypeTok{size=}\OtherTok{NULL}\NormalTok{, }\DataTypeTok{endpoint=}\OtherTok{NULL}\NormalTok{,}
                         \DataTypeTok{knots=}\DecValTok{40}\NormalTok{, }\DataTypeTok{se=}\OtherTok{TRUE}\NormalTok{, }\DataTypeTok{conf=}\FloatTok{0.95}\NormalTok{, }\DataTypeTok{nboot=}\DecValTok{50}\NormalTok{, }\DataTypeTok{details=}\OtherTok{FALSE}\NormalTok{) \{}
\NormalTok{  plot.names =}\StringTok{ }\KeywordTok{c}\NormalTok{(}\StringTok{"(a)Sample completeness profiles"}\NormalTok{,}
                 \StringTok{"(b)Size-based rarefaction/extrapolation"}\NormalTok{,}
                 \StringTok{"(c)Asymptotic and empirical diversity profiles"}\NormalTok{,}
                 \StringTok{"(d)Coverage-based rarefaction/extrapolation"}\NormalTok{,}
                 \StringTok{"(e)Evenness profiles"}\NormalTok{)}
\NormalTok{  table.names =}\StringTok{ }\KeywordTok{c}\NormalTok{(}\StringTok{"STEP1.Sample completeness profiles"}\NormalTok{,}
                  \StringTok{"STEP2.Asymptotic analysis"}\NormalTok{,}
                  \StringTok{"STEP3.Non-asymptotic coverage-based rarefaction and extrapolation analysis"}\NormalTok{,}
                  \StringTok{"STEP4:Evenness among species abundances"}\NormalTok{)}
\NormalTok{  ## 4 Details ##}
\NormalTok{  SC.table <-}\StringTok{ }\KeywordTok{SC}\NormalTok{(data, }\DataTypeTok{q=}\KeywordTok{seq}\NormalTok{(}\DecValTok{0}\NormalTok{,}\DecValTok{2}\NormalTok{,}\FloatTok{0.2}\NormalTok{), datatype, nboot, conf)}
\NormalTok{  RE.table <-}\StringTok{ }\KeywordTok{iNEXT}\NormalTok{(data, }\DataTypeTok{q=}\KeywordTok{c}\NormalTok{(}\DecValTok{0}\NormalTok{,}\DecValTok{1}\NormalTok{,}\DecValTok{2}\NormalTok{), datatype, size, endpoint, knots, se, conf, nboot)}
\NormalTok{  asy.table <-}\StringTok{ }\NormalTok{iNEXT}\OperatorTok{:::}\KeywordTok{AsymDiv}\NormalTok{(data, }\DataTypeTok{q=}\KeywordTok{seq}\NormalTok{(}\DecValTok{0}\NormalTok{, }\DecValTok{2}\NormalTok{, }\FloatTok{0.2}\NormalTok{), datatype, nboot, conf)}
\NormalTok{  ## Evenness ##}
\NormalTok{  estD =}\StringTok{ }\KeywordTok{estimateD}\NormalTok{(data, }\DataTypeTok{q=}\KeywordTok{seq}\NormalTok{(}\DecValTok{0}\NormalTok{,}\DecValTok{2}\NormalTok{,}\FloatTok{0.1}\NormalTok{), datatype, }\DataTypeTok{base=}\StringTok{"coverage"}\NormalTok{, }\DataTypeTok{level=}\OtherTok{NULL}\NormalTok{, }\DataTypeTok{nboot=}\DecValTok{0}\NormalTok{)}
\NormalTok{  maxC =}\StringTok{ }\KeywordTok{min}\NormalTok{(}\KeywordTok{unique}\NormalTok{(estD[,}\StringTok{"SC"}\NormalTok{]))}
  \CommentTok{# est = estimateD(data, q=seq(0,2,0.2), datatype, base="coverage", level=maxC, nboot=0)}
\NormalTok{  even.table =}\StringTok{ }\KeywordTok{cbind}\NormalTok{(estD[,}\KeywordTok{c}\NormalTok{(}\StringTok{"site"}\NormalTok{,}\StringTok{"order"}\NormalTok{)],}
                     \DataTypeTok{Evenness=}\KeywordTok{as.numeric}\NormalTok{(}\KeywordTok{sapply}\NormalTok{(}\KeywordTok{unique}\NormalTok{(estD}\OperatorTok{$}\NormalTok{site), }\ControlFlowTok{function}\NormalTok{(k) \{}
\NormalTok{                       tmp=(estD }\OperatorTok\StringTok{ }\KeywordTok{filter}\NormalTok{(site}\OperatorTok{==}\NormalTok{k))}\OperatorTok{$}\NormalTok{qD; tmp}\OperatorTok{/}\NormalTok{tmp[}\DecValTok{1}\NormalTok{]\}))}
\NormalTok{  )}

\NormalTok{  level =}\StringTok{ }\KeywordTok{levels}\NormalTok{(RE.table}\OperatorTok{$}\NormalTok{DataInfo}\OperatorTok{$}\NormalTok{site)}
\NormalTok{  SC.table}\OperatorTok{$}\NormalTok{Site =}\StringTok{ }\KeywordTok{factor}\NormalTok{(SC.table}\OperatorTok{$}\NormalTok{Site, level)}
\NormalTok{  asy.table}\OperatorTok{$}\NormalTok{Site =}\StringTok{ }\KeywordTok{factor}\NormalTok{(asy.table}\OperatorTok{$}\NormalTok{Site, level)}
\NormalTok{  even.table}\OperatorTok{$}\NormalTok{site =}\StringTok{ }\KeywordTok{factor}\NormalTok{(even.table}\OperatorTok{$}\NormalTok{site, level)}

\NormalTok{  ## 5 figures ##}
\NormalTok{  SC.plot <-}\StringTok{ }\KeywordTok{ggSC}\NormalTok{(SC.table) }\OperatorTok{+}
\StringTok{    }\KeywordTok{labs}\NormalTok{(}\DataTypeTok{title=}\NormalTok{plot.names[}\DecValTok{1}\NormalTok{]) }\OperatorTok{+}
\StringTok{    }\KeywordTok{theme}\NormalTok{(}\DataTypeTok{text=}\KeywordTok{element_text}\NormalTok{(}\DataTypeTok{size=}\DecValTok{10}\NormalTok{),}
          \DataTypeTok{plot.margin =} \KeywordTok{unit}\NormalTok{(}\KeywordTok{c}\NormalTok{(}\FloatTok{5.5}\NormalTok{,}\FloatTok{5.5}\NormalTok{,}\FloatTok{5.5}\NormalTok{,}\FloatTok{5.5}\NormalTok{), }\StringTok{"pt"}\NormalTok{))}
\NormalTok{  size.RE.plot <-}\StringTok{ }\KeywordTok{ggiNEXT}\NormalTok{(RE.table, }\DataTypeTok{type=}\DecValTok{1}\NormalTok{, }\DataTypeTok{facet.var=}\StringTok{"order"}\NormalTok{, }\DataTypeTok{color.var=}\StringTok{"order"}\NormalTok{) }\OperatorTok{+}
\StringTok{    }\KeywordTok{labs}\NormalTok{(}\DataTypeTok{title=}\NormalTok{plot.names[}\DecValTok{2}\NormalTok{]) }\OperatorTok{+}
\StringTok{    }\KeywordTok{theme}\NormalTok{(}\DataTypeTok{text=}\KeywordTok{element_text}\NormalTok{(}\DataTypeTok{size=}\DecValTok{10}\NormalTok{),}
          \DataTypeTok{plot.margin =} \KeywordTok{unit}\NormalTok{(}\KeywordTok{c}\NormalTok{(}\FloatTok{5.5}\NormalTok{,}\FloatTok{5.5}\NormalTok{,}\FloatTok{5.5}\NormalTok{,}\FloatTok{5.5}\NormalTok{), }\StringTok{"pt"}\NormalTok{))}
\NormalTok{  cover.RE.plot <-}\StringTok{ }\KeywordTok{ggiNEXT}\NormalTok{(RE.table, }\DataTypeTok{type=}\DecValTok{3}\NormalTok{, }\DataTypeTok{facet.var=}\StringTok{"order"}\NormalTok{, }\DataTypeTok{color.var=}\StringTok{"order"}\NormalTok{) }\OperatorTok{+}
\StringTok{    }\KeywordTok{labs}\NormalTok{(}\DataTypeTok{title=}\NormalTok{plot.names[}\DecValTok{4}\NormalTok{]) }\OperatorTok{+}
\StringTok{    }\KeywordTok{theme}\NormalTok{(}\DataTypeTok{text=}\KeywordTok{element_text}\NormalTok{(}\DataTypeTok{size=}\DecValTok{10}\NormalTok{),}
          \DataTypeTok{plot.margin =} \KeywordTok{unit}\NormalTok{(}\KeywordTok{c}\NormalTok{(}\FloatTok{5.5}\NormalTok{,}\FloatTok{5.5}\NormalTok{,}\FloatTok{5.5}\NormalTok{,}\FloatTok{5.5}\NormalTok{), }\StringTok{"pt"}\NormalTok{))}
\NormalTok{  asy.plot <-}\StringTok{ }\KeywordTok{ggAsymDiv}\NormalTok{(asy.table) }\OperatorTok{+}
\StringTok{    }\KeywordTok{labs}\NormalTok{(}\DataTypeTok{title=}\NormalTok{plot.names[}\DecValTok{3}\NormalTok{]) }\OperatorTok{+}
\StringTok{    }\KeywordTok{theme}\NormalTok{(}\DataTypeTok{text=}\KeywordTok{element_text}\NormalTok{(}\DataTypeTok{size=}\DecValTok{10}\NormalTok{),}
          \DataTypeTok{plot.margin =} \KeywordTok{unit}\NormalTok{(}\KeywordTok{c}\NormalTok{(}\FloatTok{5.5}\NormalTok{,}\FloatTok{5.5}\NormalTok{,}\FloatTok{5.5}\NormalTok{,}\FloatTok{5.5}\NormalTok{), }\StringTok{"pt"}\NormalTok{))}
\NormalTok{  even.plot <-}\StringTok{ }\KeywordTok{ggEven}\NormalTok{(even.table) }\OperatorTok{+}
\StringTok{    }\KeywordTok{labs}\NormalTok{(}\DataTypeTok{title=}\NormalTok{plot.names[}\DecValTok{5}\NormalTok{]) }\OperatorTok{+}
\StringTok{    }\KeywordTok{theme}\NormalTok{(}\DataTypeTok{text=}\KeywordTok{element_text}\NormalTok{(}\DataTypeTok{size=}\DecValTok{10}\NormalTok{),}
          \DataTypeTok{plot.margin =} \KeywordTok{unit}\NormalTok{(}\KeywordTok{c}\NormalTok{(}\FloatTok{5.5}\NormalTok{,}\FloatTok{5.5}\NormalTok{,}\FloatTok{5.5}\NormalTok{,}\FloatTok{5.5}\NormalTok{), }\StringTok{"pt"}\NormalTok{))}

\NormalTok{  ##  Outpue_summary ##}
\NormalTok{  summary =}\StringTok{ }\KeywordTok{list}\NormalTok{(}\KeywordTok{summary.deal}\NormalTok{(SC.table, }\DecValTok{1}\NormalTok{),}
                 \KeywordTok{summary.deal}\NormalTok{(asy.table, }\DecValTok{2}\NormalTok{),}
                 \KeywordTok{summary.deal}\NormalTok{(estD, }\DecValTok{3}\NormalTok{),}
                 \KeywordTok{summary.deal}\NormalTok{(even.table, }\DecValTok{4}\NormalTok{, estD)}
\NormalTok{  )}
  \KeywordTok{names}\NormalTok{(summary) =}\StringTok{ }\NormalTok{table.names}

\NormalTok{  ##  Output_figures ##}
  \CommentTok{# steps.plot = grid.arrange(SC.plot, size.RE.plot, asy.plot,}
  \CommentTok{#                          cover.RE.plot, even.plot, nrow=2)}
\NormalTok{  steps.plot =}\StringTok{ }\KeywordTok{ggarrange}\NormalTok{(SC.plot, size.RE.plot, asy.plot,}
\NormalTok{                         cover.RE.plot, even.plot}
\NormalTok{  )}
  \ControlFlowTok{if}\NormalTok{ (details}\OperatorTok{==}\OtherTok{FALSE}\NormalTok{) \{}
\NormalTok{    ans <-}\StringTok{ }\KeywordTok{list}\NormalTok{(}\DataTypeTok{summary =}\NormalTok{ summary, }\DataTypeTok{figure =}\NormalTok{ steps.plot)}
\NormalTok{  \} }\ControlFlowTok{else}\NormalTok{ \{}
\NormalTok{    tab =}\StringTok{ }\KeywordTok{list}\NormalTok{(}\StringTok{"Sample Completeness"}\NormalTok{ =}\StringTok{ }\NormalTok{SC.table, }\StringTok{"iNEXT"}\NormalTok{ =}\StringTok{ }\NormalTok{RE.table,}
               \StringTok{"Asymptotic Diversity"}\NormalTok{ =}\StringTok{ }\NormalTok{asy.table, }\StringTok{"Evenness"}\NormalTok{ =}\StringTok{ }\NormalTok{even.table)}
\NormalTok{    ans <-}\StringTok{ }\KeywordTok{list}\NormalTok{(}\DataTypeTok{summary =}\NormalTok{ summary, }\DataTypeTok{figure =}\NormalTok{ steps.plot, }\DataTypeTok{details =}\NormalTok{ tab)}
\NormalTok{  \}}
  \KeywordTok{return}\NormalTok{(ans)}
\NormalTok{\}}
\end{Highlighting}
\end{Shaded}


\end{document}
